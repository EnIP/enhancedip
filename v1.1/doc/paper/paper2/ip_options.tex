\section{On the Use of IP Options}
The choice to use IP options in EnIP was challenging.  During
the early stages of experimentation, IP options seemed an obvious 
choice for creating extensions to IP. However,
there were problems.  Hidell et. al. discuss the introduction
of \textbf{fast path and slow path} to routers.\cite{router02}.
With fast path, the line cards use a copy of the routing table to
make local forwarding decisions to other line cards based on table
lookups.  The fast path does not usually include the ability to
process packets containing IP options\cite{router01}.  As a result,
\textit{packets containing IP options are forwarded to the slow path}
CPU for processing and forwarding to the correct line card.

A few packets traversing the slow path should not cause saturation
of a router's CPU.  However, if EnIP were to become popular, a 
few test packets could become billions overnight and slow path CPU 
saturation could become a significant issue.  
The IP option used for EnIP has a fixed length of
12 bytes and the first byte is always 0x9a.  Presently, forwarding 
routers detect if IP options are present and send packets to the slow
path.  For EnIP packets, it would be necessary for the fast path to detect if IP
options are present, then determine if the first byte is 0x9a.  If so,
the packet is an EnIP packet and could then be simply forwarded based on 
the destination IP address using the line card's routing table.
According to Hidell et. al. routers have already moved
towards an architecture where there is flexibility in the fast path,
so this should not be difficult. 

It is known that ping packets with IP options are likely to traverse a router's slow
path. Rossi and Welzl conducted ping measurements to hosts
across the Internet and their results suggested a 10\% RTT increase
in a 2002 study, a 7\% RTT increase in a 2003 study\cite{ipopts02}
and a 26\% RTT increase in a 2004 study\cite{ipopts03}.
Fonseca et. al. studied the survivability of packets
containing IP options in an Origin AS, Transit AS, or Destination
AS\cite{ipopts01}.  Results showed packets with IP options that were
dropped, depended on the option used.
Between 85\% and 92\% of the drops occurred at an edge autonomous
system.  They showed that support for IP options in the wide area
could be restored, discovering the \textbf{core of the network
drops very few packets with options} with the majority of 
drops occurring in edge AS networks.

EnIP and IPv6 both require upgrades to the fast path implementations
of routers.  A major advantage of the EnIP approach is the
simplicity of the upgrade in comparison to the IPv6 core network
upgrade.  EnIP's fast path upgrade has the following advantages:
\begin{enumerate}
 \item No equipment reconfiguration is required (e.g. IPv6 address
 assignment, BGP configuration) 
 \item Providers can \textbf{independently} upgrade their fast paths.
 \item No updates to the IPv4 routing table are required
\end{enumerate}
