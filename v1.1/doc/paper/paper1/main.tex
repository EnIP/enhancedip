%% bare_jrnl.tex
% Sams
\documentclass[letterpaper,10pt]{IEEEtranTCOM}
\usepackage{url}
\usepackage{graphicx}
\usepackage{color}
\usepackage{colortbl}
\usepackage{float}
% For reading with less pages wasted uncomment the next line
%\usepackage[margin=0.5in, paperwidth=8.5in, paperheight=11 in]{geometry}
%\usepackage[margin=0.5in]{geometry}
% For doublespacing uncomment the next 2 lines
\usepackage{setspace}   
%\doublespacing   
%\floatstyle{boxed} 
\floatstyle{ruled} 
\restylefloat{figure}
\normalsize
\hyphenation{op-tical net-works semi-conduc-tor}


\begin{document}
\title{Survey of Deployed Methods\\for Increasing Internet Address Space}

\author{William~Chimiak,~\IEEEmembership{Senior Member,~IEEE,}\\
    Samuel~Patton, \\and~Stephen~Janansky}

%\author{William~Chimiak,~\IEEEmembership{Senior Member,~IEEE,}\\ Samuel~Patton,\\ and~Stephen~Janansky}


\date{December 14, 2011}

% The paper headers
%\markboth{IEEE Computer}%
%{Submitted paper}

% make the title area
\maketitle

\begin{abstract}
%\begin{figure*}
%\end{figure*}
This paper provides the background for an extension to IPv4 called 
Enhanced IP (EnIP).  EnIP extends the address space of IPv4 in 
response to the IPv4 address depletion problem. This paper serves as 
a literature survey on IPv4, IPv6, and Carrier Grade NATs.  This survey was
used to inform research on Enhanced IP, the details of which are 
covered in a seperate paper.

\end{abstract}


% IEEEtran.cls defaults to using nonbold math in the Abstract.
% This preserves the distinction between vectors and scalars. However,
% if the journal you are submitting to favors bold math in the abstract,
% then you can use LaTeX's standard command \boldmath at the very start
% of the abstract to achieve this. Many IEEE journals frown on math
% in the abstract anyway.

% Note that keywords are not normally used for peerreview papers.
%\begin{IEEEkeywords}
%IPv4 address depletion, IPv6, Enhanced IP.
%\end{IEEEkeywords}

% For peer review papers, you can put extra information on the cover
% page as needed:
% \ifCLASSOPTIONpeerreview
% \begin{center} \bfseries EDICS Category: 3-BBND \end{center}
% \fi
% For peerreview papers, this IEEEtran command inserts a page break and
% creates the second title. It will be ignored for other modes.
%% \IEEEpeerreviewmaketitle

\section{Introduction}

The rapid increase of mobile devices has created a large demand for
IP addresses.  At present, there are approximately 7 billion people and 3 billion usable IPv4 addresses.
It would be useful to have a unique address for each mobile device so that any device could call another.  
Presently this is not possible.  Since 1999, IPv6 has presented a similar
argument that a longer address would make VoIP applications simpler to deploy.  We analyzed IPv6 to learn
how it works.

In addition to IPv6 analysis, we also studied transitions during the life of IPv4.  On at least four occasions, IPv4
evolved to remedy address space exhaustion issues.  Further, we looked into the history of ARPANET
at a fundamental transition in its history: the switchover from NCP to TCP/IP.  Finally, we
performed a literature survey on Carrier Grade NAT technology.  Carrier Grade NAT can offer an immediate fix
to ISPs with limited IP address space.  Deployment of Carrier Grade NAT appears to have some additional costs in terms
of additional expense involved in law enforcement monitoring.  CGNs may also ossify the network to make future deployment
of IPv6 more difficult.

\section{Outline}
Section III of the paper discusses several transitions in the Internet Protocol
suite.  These transitions include NCP to TCP/IP, the evolution of IPv4, and IPv6.
The final section includes closing remarks.

\section {A Survey of Layer 3 Protocol Transitions}
%\paragraph{}
The following section gives an overview of transitions within three
major Internet protocols: NCP to TCP, transitions during the life
of IPv4, and the current transition involving IPv6. More details
are covered in another IEEE Computer article: Internet Protocol Literature Survey.

\begin{enumerate}
 \item 
\textbf{NCP to TCP:}
%\subsection{NCP to TCP}
NCP was a protocol that preceded TCP/IP as a transport protocol
on the ARPANET.  The plan was to begin the transition on Jan 1,
1982 and end the transition with all hosts running TCP/IP on Jan
1, 1983.  Not all sites were preparing to convert to TCP/IP, so
the TCP/IP team \textbf{turned off NCP} on the ARPANET IMP's for a
full day in mid-1982 \cite{Imp01} and for two days later that fall.
Most sites made the transition, however, a few that did not prepare
were offline for as long as three months after the switchover.

%\subsection{IPv4 Classless Inter Domain Routing (CIDR)}
\item
\textbf{IPv4 Classless Inter Domain Routing (CIDR):}
RFC 791 introduced classful addressing.  Around 1993, RFC 1517
documented classful addressing problems\cite{rfc1517}. The IETF
developed Variable Length Subnet Masks (VLSM) with Classless Inter
Domain Routing (CIDR) in RFCs 1518, 1519, and 4632.  This introduced
routing protocols that could work on arbitrary bit boundaries.
This was a benefit to network users and providers.

%\subsection{IPv4 and Network Address Translation (NAT)}
\item
\textbf{IPv4 and Network Address Translation (NAT):} NAT was another
major change in IPv4 that increased usable IP address space.
It is Network Address Port Translation
(NAPT) or PAT or IP masquerading that is the most widely used and
is used in this sense throughout this paper.  NAT allows hosts to
reuse one or more public IP addresses by rewriting packets that
traverse the NAT to appear as though they originate from the NAT's
external interface by keeping a table of port and IP mappings for
returned packets.  More information on NAT and three approaches to
implementing Carrier Grade NAT are described in greater detail in
the previous IEEE Computer Magazine paper.

%\subsection{IPv6}
\item
\textbf{IPv6:}
There is no lack of publication on IPv6.  There are 292 RFC documents
with the word IPv6 in the title.  It is difficult to determine which
of these documents represents the leading thoughts on IPv6 and which
documents can be safely ignored.  However, the abrupt transition
is not possible. For example, RFC 6343 gives some guidelines for
6to4 Deployment for a transition.
\end{enumerate}
  %sam

%\include makes a pagebreak, input does not
%\section{Conclusion}
This paper described the operation of an extension to IP called Enhanced IP or EnIP.
EnIP provides a solution to the IPv4 address depletion problem and
includes an integration plan, much of which is based on the idea that
EnIP can extend IPv4 instead of replacing it.  The authors believe the integration plan
is feasible and based on sound economic principles.

There is work to be done.  This includes evaluation of EnIP by other parties, 
consideration of other protocols, 
development of EnIP upgrades for a large variety of operating systems, as well as evaluation of the
performance and security of existing EnIP implementations.  Much further development and analysis
is needed.  The interested reader might also consider reading the work done in IPv4+4\cite{Stand04,Stand05},
as many of the design goals in this paper are similar to EnIP.

\section{Conclusion}
This paper described the operation of an extension to IP called Enhanced IP or EnIP.
EnIP provides a solution to the IPv4 address depletion problem and
includes an integration plan, much of which is based on the idea that
EnIP can extend IPv4 instead of replacing it.  The authors believe the integration plan
is feasible and based on sound economic principles.

There is work to be done.  This includes evaluation of EnIP by other parties, 
consideration of other protocols, 
development of EnIP upgrades for a large variety of operating systems, as well as evaluation of the
performance and security of existing EnIP implementations.  Much further development and analysis
is needed.  The interested reader might also consider reading the work done in IPv4+4\cite{Stand04,Stand05},
as many of the design goals in this paper are similar to EnIP.


\bibliographystyle{plain}
\bibliography{ourrefs,rfc}

\end{document}
