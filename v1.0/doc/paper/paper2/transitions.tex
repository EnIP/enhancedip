\section {A Survey of Layer 3 Protocol Transitions}
%\paragraph{}
The following section gives an overview of transitions within three
major Internet protocols: NCP to TCP, transitions during the life
of IPv4, and the current transition involving IPv6. More details
are covered in another IEEE Computer article: Internet Protocol Literature Survey.

\begin{enumerate}
 \item 
\textbf{NCP to TCP:}
%\subsection{NCP to TCP}
NCP was a protocol that preceded TCP/IP as a transport protocol
on the ARPANET.  The plan was to begin the transition on Jan 1,
1982 and end the transition with all hosts running TCP/IP on Jan
1, 1983.  Not all sites were preparing to convert to TCP/IP, so
the TCP/IP team \textbf{turned off NCP} on the ARPANET IMP's for a
full day in mid-1982 \cite{Imp01} and for two days later that fall.
Most sites made the transition, however, a few that did not prepare
were offline for as long as three months after the switchover.

%\subsection{IPv4 Classless Inter Domain Routing (CIDR)}
\item
\textbf{IPv4 Classless Inter Domain Routing (CIDR):}
RFC 791 introduced classful addressing.  Around 1993, RFC 1517
documented classful addressing problems\cite{rfc1517}. The IETF
developed Variable Length Subnet Masks (VLSM) with Classless Inter
Domain Routing (CIDR) in RFCs 1518, 1519, and 4632.  This introduced
routing protocols that could work on arbitrary bit boundaries.
This was a benefit to network users and providers.

%\subsection{IPv4 and Network Address Translation (NAT)}
\item
\textbf{IPv4 and Network Address Translation (NAT):} NAT was another
major change in IPv4 that increased usable IP address space.
It is Network Address Port Translation
(NAPT) or PAT or IP masquerading that is the most widely used and
is used in this sense throughout this paper.  NAT allows hosts to
reuse one or more public IP addresses by rewriting packets that
traverse the NAT to appear as though they originate from the NAT's
external interface by keeping a table of port and IP mappings for
returned packets.  More information on NAT and three approaches to
implementing Carrier Grade NAT are described in greater detail in
the previous IEEE Computer Magazine paper.

%\subsection{IPv6}
\item
\textbf{IPv6:}
There is no lack of publication on IPv6.  There are 292 RFC documents
with the word IPv6 in the title.  It is difficult to determine which
of these documents represents the leading thoughts on IPv6 and which
documents can be safely ignored.  However, the abrupt transition
is not possible. For example, RFC 6343 gives some guidelines for
6to4 Deployment for a transition.
\end{enumerate}
