\section{EnIP Integration}
Currently, mobile devices are the largest growing consumers of IP
addresses.  EnIP Integration begins by focusing on 
the steps to enable Enhanced IP for Mobile Devices.  Mobile service
providers could benefit from the ability to address any phone
with a unique IP address.  This is not possible with IPv4.
This would be possible with Enhanced IP.  It is envisioned that Enhanced
IP would enable VoIP systems to make calls based on 64-bit EnIP
addresses.

\paragraph{Phase 1: Deployment to Mobile Devices and Infrastructure}
Technologies such as LTE and WiMAX provide voice channels, previously
circuit switched, using Internet protocols (mainly Voice over IP or
VoIP).  This can stress a provider's overloaded NAT devices and usually requires
NAT traversal methods such as STUN to enable end-to-end communications.

NATs are typically limited on the number of devices they can
support based on the number of active ports used.  There are
only 65,535 ports available. So at best one IP can only sustain that
many connections for devices it serves.  A large amount of loaded
persistent connections can degrade NAT service.  Voice channels are
sensitive to these degradations.  EnIP resolves these for providers
due to it's stateless nature and ability to allow for end-to-end
communication.  EnIP on mobile devices is the proposed first stage of
integration.

EnIP's deployment to mobile devices should be straightforward
for all stakeholders.  For simplicity, consider the Android Open
Source Platform.  Given Android's Linux core, EnIP patches written
against the Linux Kernel are merged into a release of Android.
Device manufacturers provide carriers with their updated Android
device firmware release.  Service providers utilize their established
firmware update mechanism, updating their customers'
devices.  Legacy NAT traversal techniques are used with old
IPv4 devices for which an EnIP update is not available.  After a kernel
update the new devices are ready to communicate over EnIP.

The next step in phase 1 is for a service provider to upgrade their networks
to support EnIP.  To do this, providers first apply the EnIP NAT
patch to their NAT devices.  This involves a small modification
of the NAT kernel moduel (nf\_nat.ko) and the addition of another (eipnat.ko). 
We have demonstrated it is possible to perform this update without rebooting a
NAT server.  Next, providers update
their intermediate connections between provider networks to allow
for the passing of IP Option 26.  In addition to this, router fast paths will 
need to be updated so that IP Option 26 is no longer passed via the slow path.
Once completed, mobile devices can
perform device-to-device communication using EnIP.  EnIP connections
do not deplete the number of connections available on a NAT as
is the case with current VOIP architectures that utilize IPv4 and NAT.

It is possible for EnIP to be a successful protocol completely within the 
confines of mobile networks.  It is envisioned that mobile operators will use
EnIP to create end-to-end VoIP systems.

\paragraph{Phase 2: Deployment to Content Providers and Infrastructure}
A move of mobile devices to EnIP initially benefits mobile
providers by making simpler VoIP architectures possible.  The remaining traffic
transitting their NATs is data traffic to and from content providers.  In particular,
the large content providers produce most of the traffic.
The steps for content providers to support
EnIP are simple enough that this integration is realistic.

Content providers upgrade in two steps:  First, they integrate EnIP
into their networks.  IP Option 26 must be allowed to traverse
their networks and the fast paths of their routers must be upgraded to
quickly process EnIP packets.
All routers MUST process EnIP packets in the fast path.
Network equipment vendors should make these fast path upgrades available.  
Vendors not providing an upgrade place themselves at a competitive disadvantage.
Since many of these vendors have the experience of designing fast path
upgrades for IPv6, the minor changes required for EnIP integration should be simple.
Similarly, NAT implementations should begin to integrate EnIP.  It will also be 
necessary to upgrade firewall software in some cases so it can forward packets 
containing IP Option 26.  With this completed,
traffic can successfully traverse between content providers and
mobile customer networks.  Since the customers will already be
able to communicate over EnIP, and the packets can now flow between
provider and customer, it will be time to upgrade the servers which
provide the content.

The second step in Phase 2 is to perform an upgrade to the servers
and software which provide the content.  EnIP patches for the host
operating systems need to be deployed.  These patches are minimal and
comparable to the application of small security patches.  In Linux,
for example, it will require short patches of approximately 700
lines of code to the system's kernel in order to allow it to properly
process EnIP connections.  Once this is completed and
verified, applications need to be tested.  Our initial application
tests worked without modification(ssh, samba, apache, firefox).  
Once applications are validated,
content providers switch to the integrated EnIP-IPv4 setup allowing
them to serve both legacy IPv4 customers and the newer EnIP customer set.

\paragraph{Phase 3: Deployment to Last Mile and Home Users}
First, ISPs integrate EnIP into their core and edge devices, as
the content providers did.  Then they upgrade the Customer Premise
Equipment when additional address space is needed.  In cases where
the users own their CPEs, the vendors provide firmware updates and
users upgrade their devices.  For those who have ISP provided CPEs, the
ISP can patch the devices' NATs to support EnIP.  As before, the
patches are small so ISPs can upgrade their customer devices without
rebooting the devices.  At this point, mobile devices will be capable
of utilizing their home network for backhaul to Mobile Providers.

With the user's home network capable of passing EnIP, users could
start to make use of the protocol on their computer systems
as well.  Patches to the host OS could be provided through
auto-update procedures to all major Operating Systems.  At this
point, home users could opt-in to using EnIP on their home network,
if they wish.  Some mobile service providers who also offer home Internet services
may be motivated to upgrade some home users CPE devices early so that mobile device
voice calls can be offloaded to a wireline connection.
