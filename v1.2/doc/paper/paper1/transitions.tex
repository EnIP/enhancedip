\nopagebreak
\section {A Survey of Layer 3 Protocol Transitions}
%\paragraph{}
The following section discusses transitions within three major
protocols of the Internet Protocol Suite. The transitions discussed
are NCP to TCP/IP, IPv4 Classless Numbering, Network Address
Port Translation (NAPT usually called NAT), Carrier Grade NAT,
and the current transition involving IPv6.

\subsection{NCP to TCP/IP}
NCP was a protocol that preceded TCP/IP as a transport protocol
on the ARPANET.  In November 1981, Jon Postel wrote RFC 801, which
described the NCP to TCP/IP transition plan.  The goal was to eradicate
NCP's usage on the ARPANET and replace it with TCP/IP.  The plan was
to begin the transition on Jan 1, 1982 and end the transition with
all hosts running TCP/IP on Jan 1, 1983.  During the transition
there would be hosts running TCP/IP, hosts running NCP only, and
hosts running both NCP and TCP/IP.  Hosts running NCP and TCP/IP
were called ``dual protocol hosts'', which is similar to language
used in the IPv6 transition standards for hosts that are dual stack,
running both IPv4 and IPv6.  The transition from NCP to TCP/IP is sometimes referred to as a flag
day because there was a date when the transition was to be finished.
At the end of the transition, the IMPs would be updated with a
simple patch to stop passing NCP.  Not all sites were preparing to
convert to TCP/IP, ``so Cerf, Postel, and the TCP/IP team turned off
the NCP network channel numbers on the ARPANET IMP's for a full day
in mid-1982''\cite{Imp01}.  They also turned off NCP for two days
later that fall.  The full switchover to TCP/IP occurred on Jan 1,
1983 without too many problems although a few sites were down for
as long as three months while they upgraded their systems.

\subsection{IPv4}
Originally, the 4 byte IP address was divided into the
\textit{network number field} (the most significant 8 bits) and the
\textit{rest field}(the lower 24 bits).  This original addressing
system only allowed for 254 different networks on the Internet which
lead to the introduction of classful network addressing in 1981
as part of the Internet Protocol (IP), RFC 791.  Classful network
addressing created three types of networks: \textit{class A},
\textit{class B}, and \textit{class C}.  Two other networks were
defined later (\textit{class D and E}).  Around 1993, RFC 1517
documented two problems caused by classful addressing.  Namely,
class C networks allowed for a network to have up to 254 hosts
while class B networks could have up to 65534 hosts.  For many
organizations, 254 addresses was not enough and 65534 was too many.
Under the classful system, organizations received a class B and
many of the addresses remained unused.  A second problem caused by
classful addressing was a large increase in the routing table size
caused by advertising many geographically dispersed class C networks.
There was no opportunity for route aggregation.  It was determined
that a change was necessary, otherwise there would be an exhaustion
of IP network numbers.  The next major evolution was to add variable
length subnet masks (VLSM).  VLSM allows networks to be divided
into smaller sizes than was possible under the classful system.  
Classless routing also introduced routing protocols that could work on arbitrary
bit boundaries.  These included RIP-2, EIGRP, IS-IS, and OSPF.

Another major change in IPv4's usage came with the introduction of 
NAT. While there are many uses of NAT, this
paper specifically refers to NAPT, which is also referred to as PAT
or IP masquerading.  RFC 2663 clarifies many terms in this 
mechanism. NAPT is the most widely used form of NAT and
the form of NAT discussed throughout this paper.  NAT allows one
or more hosts to reuse one or more public IP addresses by rewriting
packets that traverse the NAT to appear as though they originate from
the NAT's external interface.  The NAT keeps a table of port and
IP mappings so returned packets can be sent back to the internal
host that originated the connection.  There are some limits to
NAT devices.  A well understood limitation is that since end-to-end
connectivity is broken, protocols which embed their addresses in a
packet's payload are broken without additional fixup modules in the
NAT device.  Protocols such as FTP and SIP are known to have problems
traversing NATs without additional effort.   A second
problem with NATs involves the limits of the multiplexing mechanism.
There are 65535 TCP ports, thus, a quick analysis might suggest that
as many as 65535 concurrent TCP connections per public NAT IP address
is possible.  However, Miyakawa mentions an upper limit of 32000
concurrent TCP connections per IP\cite{Apnic03}.  We found a Linux
IP masquerading reference which mentions a default limit of 4096
connections and an upper limit of 32000 connections if the variables
\textbf{PORT\_MASQ\_BEGIN} and \textbf{PORT\_MASQ\_END} are adjusted
in \textit{/usr/src/linux/net/ipv4/ip\_masq.h}.  However,
this reference was only relevant to 2.4 linux kernels and at the time
of this writing 2.6 and 3.0 kernels represent modern Linux kernels.
A quick look at a more recent kernel (2.6.38) indicates that the
previously mentioned file is no longer present.  It is not clear why
Miyakawa and the 2.4 linux kernel suggest an upper limit of 32000
connections.  Further investigation into modern NAT multiplexing
mechanisms might be warranted.  One of the more interesting aspects
of Miyakawa's work was his profiling of popular applications
for the number of concurrent TCP connections required\cite{Apnic03}.
For example, he noted that iTunes required 270 connections to be
open at one point in time.  Miyakawa's results are very interesting
at a time when some Internet providers are implementing a second
layer of NAT to increase the lifespan of their IPv4 address pools.
This second layer of NAT is referred to as Carrier Grade NAT or
Large Scale NAT.  While Miyakawa's results are very interesting,
it would be beneficial to recreate his experiments for verification.


Three approaches to implementing Carrier Grade NAT were
encountered during research: Nat444, Dual-Stack Lite, and
NAT64.  In all of these approaches,
the Carrier Grade NAT is configured with a number of public IP
addresses for use in NAT.  In the NAT 444 scenario, customer
equipment is allocated an RFC 1918 address on the WAN interface
instead of a public IPv4 address.  The traffic is NATed once at the
customer's location and a second time at the Carrier Grade NAT, where
a public IP address is assigned.  With Dual-Stack Lite the customer
is provided modified customer premises equipment.  The customer's
LAN is IPv4.  When packets reach the CPE default gateway, instead of
being NATed, the packets are encapsulated in IPv6 packets and carried
to the provider's Carrier Grade NAT device.   At the Carrier NAT the
original IPv4 packets are removed from the IPv6 packet and NATed
with a public IP address.  A state table is maintained in order
to send return packets back to the appropriate CPE IPv6 address.
NAT64 requires the customer network to be completely IPv6 and all
connections to be originated by a DNS request.  When the customer
wishes to connect to an IPv4 address, such as a web site, the
customer machine issues a AAAA DNS request, which is received by
a DNS64 server.  The DNS64 server takes the request and since it
is also connected to the IPv4 network performs an A record lookup.
Once the A record lookup is returned, the DNS64 server returns a AAAA
response embedding the A record's answer in the least signficant
32-bits of the 128-bit AAAA record response.  The most significant
96-bits assures the packet will route to the NAT64 gateway.  When the
client sends the IPv6 packet it is received by the NAT64 gateway,
a router connected to both the IPv4 and IPv6 networks.  The NAT64
gateway translates the packet into an IPv4 packet and maintains a
state table for any packets that return.  In this way, an IPv6-only
host can reach IPv4 servers.  There are other translation protocols
in use not covered here but mentioned in an attempt to be complete.
They are: IVI, gogo6, Nat66, 4rd, and Nat46.

Miyakawa's work is useful for taking a population of users and
estimating the number of public IP addresses required on a Carrier
Grade NAT to accomodate $N$ customers.  Miyakawa also provides
some insight into the economics of Carrier Grade NAT.  He notes
it will be necessary to charge users less and CGN service will be
more expensive to operate because of logging.  There are
logging complexities with DS-Lite, 
but Doyle also notes logging packets in
Nat444 scenarios may not be more expensive.  Liu et. al. confirm
the technical and economic difficulties of logging Dual-Stack
Lite and NAT64 networks\cite{Nat03}.  Another interesting aspect
of Carrier Grade NAT is that in some scenarios it may make the
transition to IPv6 more difficult.  From current research, it is
not clear how Dual-Stack Lite and NAT64 affect a future transition
to IPv6, Nat 444 breaks a number of the IPv6 transition protocols.
NAT 444 appears to be the simplest of the CGN options to employ.
It is not clear how widely deployed any of the CGN technologies
have become and whether the complexity and cost of logging will
force the market to seek alternative solutions.

\subsection{IPv6}
A group known as IPng (IP Next Generation) was chartered to
create a competition focused on finding a suitable replacement
for IPv4 in RFC 1550.  RFC 1454 discusses three main protocols in the
competition: SIP, PIP, and TUBA.  The group selected
a modified version of SIP called Simple Internet Protocol Plus
(SIPP) explained in RFC 1710.  SIPP extended the address size from 32 to
64 bits.  RFCs 1883 and 2460 reflects the decision
to increase SIPP's address size from
64 to 128 bits.  In addition to a vastly
increased address size in IPv6, RFC 2460 describes how
the IPv6 header has been optimized
for routing.  This was done by removing the checksum
from the IPv6 header.  In IPv4, every time a packet traverses a
router, the IP ttl is decremented and the checksum invalidated.
The checksum must be recomputed before the packet can be forwarded.
With IPv6, recomputing the checksum is not necessary.  However,
IPv4 proponents could argue from RFC 1141 that
there are optimized implementations of
the IP time-to-live (ttl) decrement and checksum update algorithm.
Further analysis is required but perhaps this optimization is lost
from the increased computational complexity of performing the most
specific match to find an entry in the IPv6 routing table?

IPv6 provides a much larger address space, which at the
present time seems necessary given the depletion of IANA's
usable IPv4 address pool.  However, the transition to IPv6 is
non-trivial.  Few disagree with the goal of providing a larger
address pool but many have been critical of the IPv6 transition
process\cite{Cpe01,ipv6mess}.  There are
presently 292 RFC documents with the word IPv6 in the title.  It is
difficult to determine which of these documents represents the
leading thoughts on IPv6 and which documents can be safely ignored.
The task of becoming familiar with the protocols that make up IPv6
is large.   What follows is a discussion of the IPv6 transition using
five catergories: transitioning the core, transitioning the edge,
deployment to end hosts, transitioning applications, and security.

The core of the Internet operates through a number of voluntary
agreements where Internet service providers agree to allow packets
to flow between their networks.  A large tier
1 provider such as Level 3 has over 3000 BGP peering sessions,
which represent some number of business agreements less than or
equal to 3000.  Each business agreement involves the
setup of a BGP connection between providers for exchange of routes.
Typically, this is done with phone and/or email contact to setup
the BGP session.  For the IPv6 transition, providers like Level3
must obtain an IPv6 allocation, perform address planning to assign
IPv6 addresses to all relevant interfaces of their routers, and
contact their peers to setup a BGP session for the exchange of IPv6
routes.  Their peers also must obtain an IPv6 address allocation,
perform address planning, and assign addresses to the relevant
network interfaces before the exchange of IPv6 routes can occur
\cite{Peering02}.  An attempt was
made to determine the number of IPv6 autonomous systems advertising
routes into the IPv6 routing table.  In order to do this, three
bgp monitoring sources were queried.
Each of the three sources provide different numbers for the number of
IPv4 and IPv6 autonomous systems.  The numbers found are as follows:

\begin{table}[htp]
\begin{center}
\begin{tabular}{| l | c | c | c |}
\hline
 & \textbf{Date} & \textbf{IPv4} & \textbf{IPv6}  \\
\hline
 \textbf{potaroo} & October 2011 & 25296 & 741  \\
\hline
 \textbf{bgpmon} & August 2011  & 32768 & 2628 \\
\hline
\textbf{he.net} & August 2011 & 38604 & 4471 \\
\hline
\end{tabular}
\caption{Disagreement on the number of AS's in the routing tables}
\label{tab:template}
\end{center}
\end{table}

The bgpmon numbers indicate 7\% of the IPv4 autonomous systems have
enabled IPv6 and are advertising their routes into the IPv6 routing
table.  However, the potaroo numbers indicate that .029\% of the IPv4
autonomous systems are advertising IPv6 routing information while
the Hurricane Electric numbers indicate 11.6\%.  No conclusions were
drawn based on these numbers.  A discussion on IPv6 opportunities by
Vinton Cerf to the Google IPv6 Implementer's conference references 
IPv6 peering as one of the areas where more
work needs to occur.  He suggests providers should
enable IPv6 on the interfaces where there is already IPv4 peering
and that no additional peering agreements are necessary.  Cerf has
also discussed problems with home routers supporting IPv6, indicating
there is lack of a business incentive.  Standards that describe IPv6
in CPE devices began to appear in 2010\cite{Cpe02,Broadband01}.
For example, the Broadband Forum published TR-124 in May, 2010.
This document describes functional requirements for Broadband
residential gateways.  As part of this document, 22 IETF RFCs are
listed as required for meeting IPv6 requirements\cite{Broadband01}.
The IETF also describes basic requirements for customer edge routers
in RFC 6204, which was published in August, 2011.
While a few CPE devices have supported IPv6 for a while \cite{Cpe01},
there have also been a number of problems with lack of support
or broken IPv6 support in CPE implementations\cite{Cpe01,Cpe02}.
For example, some vendors of CPE devices include 6to4 support,
which is a transition technology for reaching the IPv6 Internet by
tunneling over IPv4 to an anycasted server.  Some vendors shipped
their equipment with 6to4 enabled by default which led to a number
of problems outlined in RFC 6343.  There are thirty three 6to4
gateways on the bgpmon web site.  Little information
is available in terms of the maximum load these nodes are capable
of handling.  It is not clear whether RFC 6204 or the Broadband
Forum standards represent the correct IPv6 IETF implementation.
It is not clear, why new standards are
required for CPE devices since they are routers.

With respect to deploying IPv6 to end hosts, many modern
operating systems support device configuration via neighbor
discovery as detailed in RFC 2461 and DHCPv6 described in RFC
3315.  However, one problem is that not all operating systems 
support both neighbor discovery and DHCPv6.  
As a result, it might be
necessary to choose an address assignment strategy where some
legacy operating systems can not receive IPv6 addresses from the
automatic mechanisms.

IPv6 requires some legacy software to be updated.  For example,
if a program uses \textit{gethostbyname} for dns lookups, it
will be necessary to migrate the application to use the newer
\textit{getaddrinfo} function.  \textit{getaddrinfo} returns a
\textit{struct sockaddr\_in6} if the dns resolver gets an answer
to a AAAA lookup.  It is the job of the application to pass the
\textit{struct sockaddr\_in6} to functions like connect.  This change
is simple to make.  However, there are many applications with network
functionality that need to be upgraded to handle the IPv6 case.
It seems similar to the Y2K problem.  In Y2K there were many lines
of code to be changed.  Many of these changes were not difficult
to make.  However, when considering the universe of programs to
update, the problem seems larger.  Fortunately, the model outlined in
the IPv6 transition RFCs does not suggest the IPv4 network will go
away as explained in RFC 4213.  Instead, emphasis is placed on the importance
of the IPv4 network during the change.  This strategy is called
\textit{dual stack}, since hosts and routers will run IPv4 and IPv6
stacks at the same time.  The IPv6 transition strategy also allows
for the use of transition protocols.  The IPv6 core network will
not be fully functional in the early days.  This implies that
native IPv6 connections, or direct connections into the IPv6
Internet, will not be available to everyone.  The alternative
to a native IPv6 connection is to use a transition protocol.
Transition protocols tunnel over IPv4 to reach the IPv6 network.
Some of these protocols are:

 \begin{table}[htp]
\begin{center}
 \begin{tabular}{|l|p{4in}|}
   \hline
   \textbf{RFC \#} & \textbf{Title} \\
   \hline
   4213 & Basic Transition Mechanisms for IPv6 Hosts and Routers (6in4)\\
   \hline
   3056 & Connection of IPv6 Domains via IPv4 Clouds (6to4)\\
   \hline
   4214 & Intra-Site Automatic Tunnel Addressing Protocol (ISATAP)\\
   \hline   
   4380 & Teredo: Tunneling IPv6 over UDP through Network Address Translations (Teredo) \\
   \hline
   5569 & IPv6 Rapid Deployment on IPv4 Infrastructures (6rd)\\
   \hline
  \end{tabular}
 \caption{IPv6 Transition Protocols}
\end{center}
 \end{table}

There are problems with using 6to4, 6rd, 6in4, and ISATAP from behind
NATs.  Teredo was desgined to handle this scenario.  Unfortunately,
Hoagland's study of Teredo notes it can cause unintended security
problems since it tunnels through NATs and firewalls and places
a computer directly onto the IPv6 Internet.  This is of concern
since Teredo is on by default on Windows Vista.
The availability of Teredo tunneling may
create a disincentive for ISPs to deploy Native IPv6 and for home
users to upgrade their NAT and other perimeter devices to support
the possibility of native IPv6 connectivity.

With respect to IPv6 costs, RTI prepared a report at the behest of
the U.S. National Institute for Standards and Technology suggesting
the move to IPv6 will cost U.S. entities \$25 billion over 28
years during the period 1997 to 2025\cite{RTI01}.  In addition
to estimating costs, the study also looks at potential benefits
of switching to IPv6.  One example is an estimated cost savings
from reducing NATs and the move to end to end IP reachability,
thus enabling VOIP to gain more popularity.  They estimate that
widespread VOIP usage could result in a savings of \$7.8 billion
annually.  This study has caused controversy among IPv6 proponents
who believe the roll out cost estimates are too high.
There is also belief that not enough emphasis was placed on the
costs of not switching to IPv6.  Very little information is publicly
available on the costs of upgrading software and hardware to be
compatible with IPv6.  While the costs of licensing for operating
systems based on Windows is widely available, the costs associated
with upgrading network hardware such as switches, firewalls, and
routers is not widely available in open literature.

Running a dual-stack network can increase exposure for security
vulnerability.  Neville-Neil of the Freebsd team documented a problem
in 2007 where IPv6 was partially enabled by default on Freebsd.
By default systems came up with link local addresses.  Since many
users were not aware of this behavior, their firewall rules lacked
configuration to protect the IPv6 link local address\cite{Freebsd01}.
Neville-Neil outlined how an attacker might take advantage of this.
The solution the Freebsd developers chose was to disable IPv6
in the default configuration.  Neville-Neil also mentions the
RH0 security problem.  IPv6 includes extension headers for adding
functionality to the IPv6 header much like IPv4's IP options.  One
type of routing header is the Type 0 header.  With this mechanism,
it is possible to specify the path a packet takes by specifying
IPv6 addresses as part of the routing header.  This mechanism is
similar to IPv4's loose source routing, which was itself the source
of security issues\cite{RH01}.  The IETF later deprecated support
for RH0 in RFC 5095.  There have been security issues described for
the 6to4 transition protocol.  These issues include risks for denial
of service and increased risk of spoofing IPv6 packets described
in RFC 3964.

With respect to the LAN, problems relating to the Neighbor discovery
protocols are well documented in RFCs 4861 and 4862.  For example,
RFC 6104 shows one potential problem involves rogue router 
advertisements, whether it be from misconfiguration or from a malicious person.
The attacks on neighbor discovery are similar to the attacks on ARP.
Detailed threat models for neighbor discovery are included in RFC 3756.

IPSec is included as part of IPv6 explained in RFC 2401.  There is work
on Secure Neighbor Discovery (SEND) in RFC 3971 
to address security problems outlined in RFC 3756: \textit{IPv6
Neighbor Discovery(ND) Trust Models and Threats}.
